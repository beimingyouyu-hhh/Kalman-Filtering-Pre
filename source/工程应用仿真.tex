%================================================================
\section{工程应用与仿真}
%================================================================

\begin{frame}
    \frametitle{仿真案例:一维小车位置追踪}
    \textbf{场景假设:}
    \begin{itemize}
        \item 小车做匀速运动,速度 $v=1m/s$。
        \item 过程噪声 $\bm{Q}$:路面颠簸导致速度微小波动。
        \item 测量噪声 $\bm{R}$:GPS 定位误差 $\pm 5m$。
    \end{itemize}

\end{frame}
\begin{frame}
    \frametitle{仿真案例:一维小车位置追踪}
    % \vspace{-5cm}
    \begin{figure}[H]
        \centering
        \includegraphics[width = .5\textwidth]{Kal.png}
        \caption{利用Matlab模拟卡尔曼滤波}
    \end{figure}
\end{frame}
\begin{frame}
    \frametitle{参数 $\bm{Q}$ 与 $\bm{R}$ 的调节策略}
    在信息工程实践中,$\bm{F}$ 和 $\bm{H}$ 通常由物理系统决定,但 $\bm{Q}$ 和 $\bm{R}$ 是调试的关键。

    \begin{columns}
        \column{0.5\textwidth}
        \begin{block}{信任模型 ($\bm{Q}$ 小, $\bm{R}$ 大)}
            \begin{itemize}
                \item 滤波结果非常平滑
                \item 对突变反应滞后 (Lag)
            \end{itemize}
        \end{block}

        \column{0.5\textwidth}
        \begin{block}{信任观测 ($\bm{Q}$ 大, $\bm{R}$ 小)}
            \begin{itemize}
                \item 紧跟观测值变化
                \item 引入较多噪声 (Noise)
            \end{itemize}
        \end{block}
    \end{columns}

    \vspace{0.5cm}
    \textbf{结论:} 调参本质上是在\textbf{响应速度}与\textbf{平滑度}之间做权衡。
\end{frame}

\begin{frame}
    \frametitle{进阶:多传感器融合 (Sensor Fusion)}
    卡尔曼滤波是多传感器融合的基石。

    \textbf{案例:无人机姿态解算}
    \begin{itemize}
        \item \textbf{陀螺仪 (Gyro)}:短时精度高,长时有积分漂移。
        \item \textbf{加速度计 (Accel)}:静态精度高,动态噪声大。
        \item \textbf{融合方案}:利用 Kalman Filter 融合两者数据。
              \begin{equation*}
                  \text{Optimal Angle} = K \cdot \text{Accel} + (1-K) \cdot \text{Gyro}
              \end{equation*}
    \end{itemize}
\end{frame}

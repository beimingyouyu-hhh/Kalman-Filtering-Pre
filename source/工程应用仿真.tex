%================================================================
\section{工程应用与仿真}
%================================================================

\begin{frame}
    \frametitle{仿真案例:一维小车位置追踪}
    \textbf{场景假设:}
    \begin{itemize}
        \item 小车做匀速运动,速度 $v=1m/s$。
        \item 过程噪声 $\bm{Q}$:路面颠簸导致速度微小波动。
        \item 测量噪声 $\bm{R}$:GPS 定位误差 $\pm 5m$。
    \end{itemize}

\end{frame}
\begin{frame}
    \frametitle{仿真案例:一维小车位置追踪}
    % \vspace{-5cm}
    \begin{figure}[H]
        \centering
        \includegraphics[width = .5\textwidth]{Kal.png}
        \caption{利用Matlab模拟卡尔曼滤波}
    \end{figure}
\end{frame}
\begin{frame}
    \frametitle{参数 $\bm{Q}$ 与 $\bm{R}$ 的调节策略}
    在信息工程实践中,$\bm{F}$ 和 $\bm{H}$ 通常由物理系统决定,但 $\bm{Q}$ 和 $\bm{R}$ 是调试的关键。

    \begin{columns}
        \column{0.5\textwidth}
        \begin{block}{信任模型 ($\bm{Q}$ 小, $\bm{R}$ 大)}
            \begin{itemize}
                \item 滤波结果非常平滑
                \item 对突变反应滞后
            \end{itemize}
        \end{block}

        \column{0.5\textwidth}
        \begin{block}{信任观测 ($\bm{Q}$ 大, $\bm{R}$ 小)}
            \begin{itemize}
                \item 紧跟观测值变化
                \item 引入较多噪声
            \end{itemize}
        \end{block}
    \end{columns}

    \vspace{0.5cm}
    \textbf{结论:} 调参本质上是在\textbf{响应速度}与\textbf{平滑度}之间做权衡。
\end{frame}



%################################################################
% 案例一:信息工程 - 5G/6G 无线信道追踪
%################################################################

%------- 第一页:背景与痛点 -------
\begin{frame}
    \frametitle{信息工程案例 (1/3):高速移动通信的挑战}
    \textbf{背景:} 在高铁或车载通信场景(5G/V2X)中,终端的高速移动会产生严重的\textbf{多普勒频移 (Doppler Shift)}。
    
    \vspace{0.3cm}
    \begin{columns}
        \column{0.6\textwidth}
        \begin{block}{问题:快时变信道 (Fast Fading)}
            信道冲激响应 $\bm{h}_k$ 不再是常数,而是随时间快速变化的随机过程。
            \begin{itemize}
                \item 传统方法(如最小二乘 LS):假设信道在一段 block 内不变 $\rightarrow$ \textcolor{red}{失效}。
                \item 需求:必须利用\textbf{时间相关性}进行实时追踪。
            \end{itemize}
        \end{block}
        \column{0.4\textwidth}
        \centering
        \begin{tikzpicture}[scale=0.8]
            \draw[->] (0,0) -- (4,0) node[right] {$t$};
            \draw[->] (0,-1.5) -- (0,1.5) node[above] {$|h|$};
            \draw[blue, thick, domain=0:4, samples=100] plot (\x, {cos(300*\x) * exp(-0.2*\x) + 0.3*rand});
            \node at (2, -1.8) {\small 剧烈波动的信道增益};
        \end{tikzpicture}
    \end{columns}
\end{frame}

%------- 第二页:数学建模 -------
\begin{frame}
    \frametitle{信息工程案例 (2/3):AR(1) 动力学建模}
    为了使用 Kalman 滤波,我们需要建立状态方程。文献 \cite{komninakis2002multi} 提出利用一阶自回归模型 (AR1) 近似 Jakes 衰落模型。

    \begin{equation}
        \text{状态方程:} \quad \bm{h}_k = \alpha \bm{h}_{k-1} + \bm{w}_k
    \end{equation}

    \begin{itemize}
        \item $\bm{h}_k$: 复数信道系数(状态量)。
        \item $\alpha$: \textbf{自相关系数},由物理环境决定:
              $$ \alpha = J_0(2\pi f_d T_s) $$
              \small ($J_0$: 零阶贝塞尔函数, $f_d$: 多普勒频率, $T_s$: 符号周期)
        \item $\bm{w}_k$: 过程噪声,代表建模误差,方差 $\mathbf{Q} = 1 - |\alpha|^2$。
    \end{itemize}

    \begin{alertblock}{建模的核心思想}
        不仅仅是“估计”当前值,而是利用 $f_d$ 预测信道的“惯性”变化趋势。
    \end{alertblock}
\end{frame}

%------- 第三页:算法优势与对比 -------
\begin{frame}
    \frametitle{信息工程案例 (3/3):KF vs LS 性能对比}
    我们将卡尔曼滤波 (KF) 与传统的最小二乘 (LS) 估计算法进行对比。
    
    \textbf{观测方程:} $y_k = s_k \bm{h}_k + v_k$ ($s_k$ 为已知导频)

    \vspace{0.3cm}
    \begin{table}
        \centering
        \begin{tabular}{l|l|l}
            \toprule
            \textbf{特性} & \textbf{最小二乘 (LS)} & \textbf{卡尔曼滤波 (KF)} \\
            \midrule
            原理 & 仅基于当前观测 $y_k$ & 融合当前观测 $y_k$ + 历史预测 \\
            计算量 & 低 ($O(1)$) & 中等 ($O(N^3)$ 矩阵求逆) \\
            抗噪性 & 差 (随 SNR 线性下降) & \textbf{强 (有效滤除高斯白噪)} \\
            适用场景 & 静态/慢速环境 & \textbf{高速移动/低信噪比环境} \\
            \bottomrule
        \end{tabular}
    \end{table}

    \vspace{0.2cm}
    \textbf{结论:} KF 能够提供比 LS 高约 \textbf{3dB} 的信噪比增益,显著降低误码率 (BER)。
\end{frame}


%################################################################
% 案例二:微电子 - MEMS 传感器精度增强
%################################################################

%------- 第一页:硬件痛点 -------
\begin{frame}
    \frametitle{微电子案例 (1/3):MEMS 传感器的误差源}
    在微电子制造中,低成本 MEMS 陀螺仪(如手机中的 IMU)受限于工艺,存在两类主要误差 \cite{woodman2007introduction}:

    \begin{columns}
        \column{0.5\textwidth}
        \begin{block}{1. 角度随机游走 (White Noise)}
            \begin{itemize}
                \item 来源:电子热噪声、机械抖动。
                \item 特性:高频噪声,均值为0。
                \item \textbf{KF 对应:} 测量噪声矩阵 $\bm{R}$。
            \end{itemize}
        \end{block}

        \column{0.5\textwidth}
        \begin{block}{2. 零偏不稳定性 (Bias Instability)}
            \begin{itemize}
                \item 来源:温度漂移、应力释放。
                \item 特性:低频缓慢变化,类似 $1/f$ 噪声。
                \item \textbf{危害:} 直接积分会导致角度误差随时间 $t$ 二次发散。
            \end{itemize}
        \end{block}
    \end{columns}
\end{frame}

%------- 第二页:状态扩维建模 -------
\begin{frame}
    \frametitle{微电子案例 (2/3):状态扩维 (Augmented State)}
    为了消除零偏 (Bias),我们将未知的 Bias 也视为一个\textbf{“状态”}放入滤波器中进行估计。

    \textbf{系统状态向量:} $\bm{x} = [\theta, \ bias]^T$ 
    
    \vspace{0.3cm}
    \textbf{离散化状态空间模型:}
    \begin{equation}
        \begin{bmatrix} \theta_k \\ b_k \end{bmatrix} = 
        \begin{bmatrix} 1 & -\Delta t \\ 0 & 1 \end{bmatrix} 
        \begin{bmatrix} \theta_{k-1} \\ b_{k-1} \end{bmatrix} + 
        \begin{bmatrix} \omega_{mea} \Delta t \\ 0 \end{bmatrix} + \bm{w}_k
    \end{equation}

    \begin{itemize}
        \item 第一行:$\theta_k = \theta_{k-1} + (\omega_{mea} - b_{k-1})\Delta t$ \\
              \small (解释:真实角速度 = 测量值 - 漂移值)
        \item 第二行:$b_k = b_{k-1} + w_b$ \\
              \small (解释:假设漂移本身是一个随机游走过程,即 Bias 也是在变的)
    \end{itemize}
\end{frame}

%------- 第三页:软件定义精度 -------
\begin{frame}
    \frametitle{微电子案例 (3/3):软件定义硬件精度}
    通过卡尔曼滤波,我们实现了一种“虚拟传感器”的效果。

    \begin{columns}
        \column{0.6\textwidth}
        \textbf{算法流程闭环:}
        \begin{enumerate}
            \item \textbf{预测:} 利用上一时刻估算的 Bias 修正当前的陀螺仪读数。
            \item \textbf{观测:} 引入加速度计或磁力计数据作为“观测值” $z_k$。
            \item \textbf{校正:} 计算卡尔曼增益 $\bm{K}$,同时更新 \textbf{角度 $\theta$} 和 \textbf{零偏 $b$}。
        \end{enumerate}
        
        \column{0.4\textwidth}
        \begin{exampleblock}{工程价值}
            \centering
            \textbf{精度提升 10$\times$}
            
            \raggedright
            \small
            利用 KF 算法,可以将售价 \$5 的消费级 MEMS 芯片,提升至接近工业级 (\$50) 的性能表现。
        \end{exampleblock}
    \end{columns}

    \vspace{0.5cm}
    \textit{注:这是自动驾驶、无人机飞控中最底层的核心算法之一。}
\end{frame}

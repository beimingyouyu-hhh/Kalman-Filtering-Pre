%================================================================
\section{扩展卡尔曼滤波 (EKF)}
%================================================================

\begin{frame}
    \frametitle{1. 线性系统的局限性}
    \begin{alertblock}{标准 KF 的致命弱点}
        标准卡尔曼滤波假设系统是\textbf{线性 (Linear)} 的:
        $$ \bm{x}_k = \bm{F}\bm{x}_{k-1} \quad \text{和} \quad \bm{z}_k = \bm{H}\bm{x}_k $$
    \end{alertblock}

    \textbf{然而,现实世界充满了非线性:}
    \begin{itemize}
        \item \textbf{机器人运动}:$x_{new} = x_{old} + v \cdot \cos(\theta) \cdot \Delta t$ (包含三角函数)
        \item \textbf{雷达观测}:$r = \sqrt{x^2 + y^2}$ (包含平方根)
    \end{itemize}

    \vspace{0.2cm}
    \textbf{问题:} 高斯分布经过非线性变换后,\textbf{不再是高斯分布}(形状会扭曲),导致标准 KF 的公式失效。
\end{frame}

\begin{frame}
    \frametitle{2. 解决思路:线性化 (Linearization)}
    既然非线性太难处理,我们能不能在\textbf{局部}把它看作线性的?

    \begin{block}{泰勒级数展开 (Taylor Series Expansion)}
        对于非线性函数 $f(x)$,我们在估计点 $\hat{x}$ 附近做一阶展开:
        \begin{equation}
            f(x) \approx f(\hat{x}) + \underbrace{\frac{\partial f}{\partial x}\bigg|_{x=\hat{x}}}_{\text{切线斜率}} \cdot (x - \hat{x})
        \end{equation}
    \end{block}

    \begin{itemize}
        \item 我们抛弃高阶项,只保留一阶导数。
        \item 用\textbf{切线}来近似\textbf{曲线}。
    \end{itemize}
\end{frame}

\begin{frame}
    \frametitle{3. 核心工具:雅可比矩阵 (Jacobian Matrix)}
    在多维系统中,导数变成了\textbf{雅可比矩阵}。它是 EKF 的核心。

    假设状态转移函数为 $\bm{x}_k = f(\bm{x}_{k-1}, \bm{u}_k)$,则雅可比矩阵 $\bm{F}_k$ 为:

    \begin{equation}
        \bm{F}_k = \frac{\partial f}{\partial \bm{x}} =
        \begin{bmatrix}
            \frac{\partial f_1}{\partial x_1} & \frac{\partial f_1}{\partial x_2} & \cdots \\
            \frac{\partial f_2}{\partial x_1} & \frac{\partial f_2}{\partial x_2} & \cdots \\
            \vdots                            & \vdots                            & \ddots
        \end{bmatrix}
    \end{equation}

    \textbf{物理意义:} 描述了输入微小变化如何影响输出的每一个维度。
\end{frame}

\begin{frame}
    \frametitle{4. EKF 算法流程的变化 (对比 KF)}
    EKF 的步骤与 KF 几乎一样,区别在于\textbf{如何传递均值}和\textbf{如何传递协方差}。

    \begin{table}
        \centering
        \small
        \begin{tabular}{l|c|c}
            \toprule
            步骤             & 标准 KF (线性)                                            & 扩展 KF (非线性)                                                                         \\
            \midrule
            \textbf{状态预测}  & $\bm{x}^- = \bm{F}\bm{x}$                             & $\bm{x}^- = \mathbf{f}(\bm{x}, \bm{u})$ \textcolor{blue}{(直接代入函数)}                  \\
            \textbf{协方差预测} & $\bm{P}^- = \bm{F}\bm{P}\bm{F}^T + \bm{Q}$            & $\bm{P}^- = \mathbf{F_k}\bm{P}\mathbf{F_k}^T + \bm{Q}$ \textcolor{red}{(用雅可比)}      \\
            \midrule
            \textbf{卡尔曼增益} & $\bm{K} = \bm{P}^-\bm{H}^T(\dots)^{-1}$               & $\bm{K} = \bm{P}^-\mathbf{H_k}^T(\mathbf{H_k}\bm{P}^-\mathbf{H_k}^T + \bm{R})^{-1}$ \\
            \textbf{状态更新}  & $\bm{x} = \bm{x}^- + \bm{K}(\bm{z} - \bm{H}\bm{x}^-)$ & $\bm{x} = \bm{x}^- + \bm{K}(\bm{z} - \mathbf{h}(\bm{x}^-))$                         \\
            \bottomrule
        \end{tabular}
    \end{table}
\end{frame}

\begin{frame}
    \frametitle{5. 经典案例:雷达追踪 (Radar Tracking)}
    \begin{columns}
        \column{0.5\textwidth}
        \textbf{状态向量 (直角坐标系):} \\
        $\bm{x} = [p_x, p_y, v_x, v_y]^T$ \\
        (飞机的位置和速度)

        \vspace{0.3cm}
        \textbf{观测向量 (极坐标系):} \\
        $\bm{z} = [\rho, \phi, \dot{\rho}]^T$ \\
        (雷达测量的距离、角度、径向速度)

        \column{0.5\textwidth}
        \textbf{非线性观测函数 $h(\bm{x})$:}
        \begin{itemize}
            \item 距离:$\rho = \sqrt{p_x^2 + p_y^2}$
            \item 角度:$\phi = \arctan(p_y / p_x)$
        \end{itemize}
    \end{columns}

    \vspace{0.5cm}
    \begin{block}{问题}
        $p_x$ 和 $p_y$ 是状态,但在观测方程里被平方和求根了。这就是典型的非线性!必须求雅可比矩阵 $\bm{H}_j$。
    \end{block}
\end{frame}

\begin{frame}
    \frametitle{6. 雷达雅可比矩阵的计算}
    我们需要对 $h(\bm{x})$ 求偏导数来得到 $\bm{H}_j$:

    \small
    \begin{equation}
        \bm{H}_j = \frac{\partial h}{\partial \bm{x}} =
        \begin{bmatrix}
            \frac{\partial \rho}{\partial p_x} & \frac{\partial \rho}{\partial p_y} & 0 & 0 \\
            \frac{\partial \phi}{\partial p_x} & \frac{\partial \phi}{\partial p_y} & 0 & 0
        \end{bmatrix}
        =
        \begin{bmatrix}
            \frac{p_x}{\sqrt{p_x^2+p_y^2}} & \frac{p_y}{\sqrt{p_x^2+p_y^2}} & 0 & 0 \\
            \frac{-p_y}{p_x^2+p_y^2}       & \frac{p_x}{p_x^2+p_y^2}        & 0 & 0
        \end{bmatrix}
    \end{equation}

    \vspace{0.3cm}
    \textbf{意义:} 这个矩阵把\textbf{位置的不确定性}(直角坐标误差),投影到了\textbf{观测的不确定性}(距离和角度误差)上。
\end{frame}

\begin{frame}
    \frametitle{7. EKF 的优缺点分析}

    \begin{block}{优点 (Why use it?)}
        \begin{itemize}
            \item \textbf{事实标准}:目前导航系统(GPS/IMU 融合)、机器人定位 (SLAM) 的首选基础算法。
            \item \textbf{计算量适中}:比粒子滤波 (Particle Filter) 快得多。
        \end{itemize}
    \end{block}

    \begin{alertblock}{缺点 (Watch out!)}
        \begin{itemize}
            \item \textbf{发散风险}:如果线性化点选得不好(初始估计太差),泰勒展开误差会很大,导致滤波器发散。
            \item \textbf{繁琐的雅可比}:对于复杂的系统,手算雅可比矩阵非常痛苦且容易出错。
        \end{itemize}
    \end{alertblock}
\end{frame}

\begin{frame}
    \frametitle{8. 超越 EKF:无迹卡尔曼滤波 (UKF)}
    当非线性极其严重时,EKF 的线性化误差无法接受。

    \begin{columns}
        \column{0.6\textwidth}
        \textbf{UKF (Unscented Kalman Filter) 思路:}
        \begin{itemize}
            \item 不去线性化函数(不算雅可比)。
            \item 而是\textbf{近似概率分布}。
            \item 选取几个关键点(Sigma Points),把它们扔进非线性函数里算一算,然后再算出新的均值和方差。
        \end{itemize}

        \column{0.4\textwidth}
        \begin{block}{核心理念}
            \small
            “近似概率分布比近似非线性函数要容易得多。”
            \\ —— Julier \& Uhlmann
        \end{block}
    \end{columns}
\end{frame}
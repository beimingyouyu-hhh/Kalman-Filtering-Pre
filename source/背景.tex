%================================================================
\section{背景引入}
%================================================================

\begin{frame}
	\frametitle{现实世界的困境:不确定性}
	\begin{columns}
		\column{0.6\textwidth}
		在信息工程与控制领域,我们面临两个永恒的问题:
		\begin{itemize}
			\item \textbf{不准确的模型 (Process Noise)}:
			      \begin{itemize}
				      \item 摩擦力、风阻难以完美建模
				      \item 积分误差随时间累积(Dead Reckoning Drifting)
			      \end{itemize}
			\item \textbf{不准确的观测 (Measurement Noise)}:
			\item GPS 信号多径效应
			\item 传感器热噪声
		\end{itemize}

		\column{0.4\textwidth}
		\begin{block}{核心问题}
			当我们既不能完全相信\textbf{推算},也不能完全相信\textbf{观测}时,如何获知系统的\textbf{真实状态}?
		\end{block}
	\end{columns}
\end{frame}

\begin{frame}
	\frametitle{卡尔曼滤波 (Kalman Filter) 的诞生}
	\begin{itemize}
		\item \textbf{提出者}:Rudolf E. Kalman (1960)
		\item \textbf{核心论文}:\textit{A New Approach to Linear Filtering and Prediction Problems}
		\item \textbf{本质}:
		      \begin{itemize}
			      \item 并不是为了滤除某些频率(不同于低通/高通滤波器)。
			      \item 是一种\textbf{最优递归估计算法} (Optimal Recursive Estimator)。
			      \item 在最小均方误差 (MMSE) 准则下,对系统状态进行最优估计。
		      \end{itemize}
		\item \textbf{应用场景}:阿波罗登月导航、雷达目标追踪、导弹制导、股市预测。
	\end{itemize}
\end{frame}
%================================================================
\section{前沿探索}
%================================================================

% --- 第一部分:基于顶级综述的算法演进 (Khodarahmi et al., 2023) ---

\begin{frame}
    \frametitle{1. 算法演进全景 (Based on Top Review)}
    \small
    卡尔曼滤波已发展为庞大的家族。针对不同场景,主流算法的特性对比如下\footcite{khodarahmi_review_2023}:

    % 表格:主流算法对比
    \begin{table}
        \centering
        \renewcommand{\arraystretch}{1.2}
        \resizebox{0.95\textwidth}{!}{
            \begin{tabular}{l l l l}
                \toprule
                \textbf{算法}       & \textbf{核心机制}      & \textbf{优势 (Pros)}   & \textbf{适用场景} \\
                \midrule
                \textbf{Basic KF} & 线性递归最小二乘           & 计算极快,线性最优            & 卫星导航,稳态控制     \\
                \textbf{EKF}      & 泰勒展开线性化 ($\bm{J}$) & 工业标准,适用性广            & 机器人定位 (SLAM)  \\
                \textbf{UKF}      & 无迹变换 (Sigma点)      & 无需导数,精度更高            & 复杂非线性系统       \\
                \textbf{IMM}      & 多模型概率交互            & \textbf{搞定机动目标},平滑切换 & 导弹/无人机追踪      \\
                \bottomrule
            \end{tabular}
        }
    \end{table}
\end{frame}

% --- 第二部分:KF 与神经网络的结合 (Feng et al., 2023) ---

\begin{frame}
    \frametitle{2. 突破瓶颈:混合模型 (Hybrid Models) 的诞生}

    \begin{columns}
        \column{0.6\textwidth}
        % 左侧:放流程图(占位符)
        \centering
        \begin{figure}
            % 这里的 hybrid.png 就是您那张流程图
            % keepaspectratio 保持长宽比,width 设置为栏宽的 95%
            \includegraphics[width=0.95\textwidth, keepaspectratio]{hybrid.png}
            \caption{状态估计技术融合路线图\footcite{feng_review_2023}}
        \end{figure}

        \column{0.4\textwidth}
        % 右侧:文字解说
        \textbf{为何要融合?}
        \begin{itemize}
            \item \textbf{传统 KF}:依赖精确物理模型,难以处理未知环境。
            \item \textbf{神经网络 (NN)}:强大的数据拟合能力,但缺乏物理约束。
        \end{itemize}

        \vspace{0.3cm}
        \textbf{融合的三种模式:}
        \begin{enumerate}
            \small
            \item \textbf{串联}:互为预处理/后处理。
            \item \textbf{训练}:KF 优化 NN 权重。
            \item \textbf{辅助 (Aided)}:\textcolor{red}{\textbf{NN 实时估计 KF 参数 ($\bm{Q}, \bm{R}$)}}。
        \end{enumerate}
    \end{columns}
\end{frame}
\begin{frame}
    \frametitle{深度思考:卡尔曼滤波 vs 机器学习}
    \begin{columns}
        % 左栏:卡尔曼滤波 (Model-Based)
        \column{0.48\textwidth}
        \begin{block}{卡尔曼滤波 (KF)}
            \centering \textbf{“理性的物理学家”}
            \begin{itemize}
                \item \textbf{驱动核心}:物理模型 ($\bm{F}, \bm{H}$) + 概率统计。
                \item \textbf{透明度}:\textbf{白盒}。每一步都有明确物理含义,可解释性强。
                \item \textbf{优势}:小样本即可工作,不仅给结果,还给\textbf{置信度} ($\bm{P}$)。
                \item \textbf{劣势}:模型必须已知且准确。
            \end{itemize}
        \end{block}

        % 右栏:机器学习 (Data-Driven)
        \column{0.48\textwidth}
        \begin{exampleblock}{机器学习 (ML/NN)}
            \centering \textbf{“经验丰富的工匠”}
            \begin{itemize}
                \item \textbf{驱动核心}:海量数据 + 拟合映射。
                \item \textbf{透明度}:\textbf{黑盒}。内部权重难以解释。
                \item \textbf{优势}:能拟合极其复杂的非线性关系,无需懂物理机理。
                \item \textbf{劣势}:数据饥渴,对未见过的场景泛化能力弱。
            \end{itemize}
        \end{exampleblock}
    \end{columns}

    \vspace{0.4cm}
    \centering
    \textbf{融合的哲学:}
    \fcolorbox{blue}{white}{用 KF 的逻辑框架约束 ML 的发散,用 ML 的拟合能力弥补 KF 的模型缺陷。}
\end{frame}
% --- 第三部分:锂电池 SOC 估计应用 ---

\begin{frame}
    \frametitle{3. 落地应用:锂电池 SOC 状态估计}
    \small
    \textbf{背景:} SOC (State of Charge) 是电池管理系统的核心。估算错误会导致过充/过放,造成永久性损伤。

    \vspace{0.3cm}
    \textbf{核心技术栈 (Tech Stack):}
    \begin{columns}[t]
        \column{0.48\textwidth}
        \begin{block}{1. 非线性滤波 (NLKFs)}
            用于处理电池高度非线性的电压特性:
            \begin{itemize}
                \item \textbf{EKF / AEKF} (自适应扩展卡尔曼)
                \item \textbf{UKF / AUKF} (自适应无迹卡尔曼)
            \end{itemize}
        \end{block}

        \column{0.48\textwidth}
        \begin{block}{2. 在线参数辨识}
            实时更新电池模型参数(如内阻):
            \begin{itemize}
                \item \textbf{RLS} (递归最小二乘法)
                \item \textbf{PRBM} (多项式回归模型)
            \end{itemize}
        \end{block}
    \end{columns}

    \vspace{0.4cm}
    \textbf{实验结论:}
    在宽温域 (-5\textcelsius $\sim$ 45\textcelsius) 测试中,以下组合表现最佳:
    \begin{center}
        \fcolorbox{red}{white}{\textbf{PRBM-AUKF} \quad \& \quad \textbf{RLS-AUKF}}
    \end{center}
    \footnotesize \textit{* AUKF (Adaptive UKF) 展现了比传统 EKF 更高的精度和鲁棒性。\footcite{hossain_kalman_2022}}
\end{frame}